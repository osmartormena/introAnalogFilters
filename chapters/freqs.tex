\chapter{Resposta em frequência}

No último Capítulo, finalizamos a introdução à \ac{tf} através da sua aplicação na visualização da resposta em frequência do circuito da \fig{rc}. Neste Capítulo seremos um pouco mais amplos e gerais, porém ainda baseados em circuitos RC bastante simples.

As Figuras~\ref{fig:Hlp}, \ref{fig:Hhp}, \ref{fig:Hbp} e \ref{fig:Hsb} representam topologias simples de filtos \emph{passa"-baixas}, \emph{passa"-altas}, \emph{passa"-faixa} e \emph{rejeita"-faixa}, respectivamente. Por simplicidade, todos os valores de resistência $R$ e capacitância $C$ são idênticos.

\begin{figure}[ht]
	\centering
	\tikzsetnextfilename{Hlp}
	\begin{tikzpicture}[xscale=2.5, yscale=2]\draw
		(0,1) to[V,l_=$v_i(t)$](0,0) -- (1,0)
		(0,1) to[R,l=$R$] (1,1)
		to[C,l_=$C$,v^=$v_o(t)$] (1,0);
		\node[anchor=north] at(0.5,0) {0};
		\node[anchor=south] at(0,1) {1};
		\node[anchor=south] at(1,1) {2};
	\end{tikzpicture}
	\caption{Filtro RC passa"-baixas.}
	\label{fig:Hlp}
\end{figure}

\begin{figure}[ht]
	\centering
	\tikzsetnextfilename{Hhp}
	\begin{tikzpicture}[xscale=2.5, yscale=2]\draw
		(0,1) to[V,l_=$v_i(t)$](0,0) -- (1,0)
		(0,1) to[C,l=$C$] (1,1)
		to[R,l_=$R$,v^=$v_o(t)$] (1,0);
		\node[anchor=north] at(0.5,0) {0};
		\node[anchor=south] at(0,1) {1};
		\node[anchor=south] at(1,1) {2};
	\end{tikzpicture}
	\caption{Filtro RC passa"-altas.}
	\label{fig:Hhp}
\end{figure}

\begin{figure}[ht]
	\centering
	\tikzsetnextfilename{Hbp}
	\begin{tikzpicture}[xscale=1.5, yscale=2]\draw
		(0,1) to[V,l_=$v_i(t)$](0,0) -- (3,0)
		(0,1) to[R,l=$R$] (1,1)
		to[C,l=$C$] (2,1) -- (3,1)
		(2,1) to[R,l_=$R$,*-*] (2,0)
		(3,1) to[C,l_=$C$,v^=$v_o(t)$](3,0);
		\node[anchor=north] at(1.5,0) {0};
		\node[anchor=south] at(0,1) {1};
		\node[anchor=south] at(1,1) {2};
		\node[anchor=south] at(2.5,1) {3};
	\end{tikzpicture}
	\caption{Filtro RC passa"-faixa.}
	\label{fig:Hbp}
\end{figure}

\begin{figure}[ht]
	\centering
	\tikzsetnextfilename{Hsb}
	\begin{tikzpicture}[xscale=2, yscale=1.5]\draw
		(0,1) to[V,l_=$v_i(t)$](0,0) -- (2,0)
		(0,1) to[R,*-*,l=$R$] (1,1)
		(0,1) -- (0,2) to[C,l=$C$] (1,2) -- (1,1)
		(1,1) to[open,v^=$v_o(t)$] (1,0)
		(1,1) to[R,l=$R$] (2,1)
		to[C,l_=$C$] (2,0);
		\node[anchor=north] at(1,0) {0};
		\node[anchor=east] at(0,1.5) {1};
		\node[anchor=south west] at(1,1) {2};
		\node[anchor=south] at(2,1) {3};
	\end{tikzpicture}
	\caption{Filtro RC rejeita"-faixa.}
	\label{fig:Hsb}
\end{figure}

Lembrando que $\tau=RC$, as funções de transferência dos filtros passa"-baixa $H_{\text{PB}}(s)$, passa"-alta $H_{\text{PA}}(s)$, passa"-faixa $H_{\text{PF}}(s)$ e rejeita"-faixa $H_{\text{RF}}(s)$ são dadas por
\begin{gather}
	H_{\text{PB}}(s)=\frac{1}{\tau s+1};\\
	H_{\text{PA}}(s)=\frac{\tau s}{\tau s+1};\\
	H_{\text{PF}}(s)=\frac{\tau s}{\tau^2s^2+3\tau s+1};\\
	H_{\text{RF}}(s)=\frac{(\tau s+1)^2}{\tau^2s^2+3\tau s+1}.
\end{gather}

Como visto anteriormente, $\omega_c=1\slash\tau$. Assim, para a análise de Fourier, as respostas em frequência dos filtros são:
\begin{gather}
	H_{\text{PB}}(\omega)=\frac{1}{\displaystyle1+j\frac{\omega}{\omega_c}};\\
	H_{\text{PA}}(\omega)=\frac{\displaystyle j\frac{\omega}{\omega_c}}{\displaystyle1+j\frac{\omega}{\omega_c}};\\
	H_{\text{PF}}(\omega)=\frac{\displaystyle j\frac{\omega}{\omega_c}}{\displaystyle1-\frac{\omega^2}{\omega_c^2}+j3\frac{\omega}{\omega_c}};\\
	H_{\text{RF}}(\omega)=\frac{\displaystyle1-\frac{\omega^2}{\omega_c^2}+j2\frac{\omega}{\omega_c}}{\displaystyle1-\frac{\omega^2}{\omega_c^2}+j3\frac{\omega}{\omega_c}}.
\end{gather}

\section{O diagrama de Bode}

O diagrama de Bode representa a resposta em magnitude e em fase de uma função de transferência estável. Considerando que os sistemas sejam reais, as regras de \emph{simetria conjugada} de aplicam à \ac{tf}, assim:
\begin{itemize}
	\item a magnitude da \ac{tf} é par;
	\item a fase da \ac{tf} é ímpar.
\end{itemize}
De maneira que podemos remover as frequências negativas da representação. Ademais, mostra"-se útil utilizar um gráfico em escala logarítmica no eixo das frequências. Finalmente, o valor da resposta em magnitude é expresso em decibéis (\unit{\decibel}).

As respostas em magnitude, dadas por
\begin{gather}
	|H_{\text{PB}}(\omega)|^2=\frac{1}{\displaystyle1+\frac{\omega^2}{\omega_c^2}};\\
	|H_{\text{PA}}(\omega)|^2=\frac{\displaystyle\frac{\omega^2}{\omega_c^2}}{\displaystyle1+\frac{\omega^2}{\omega_c^2}};\\
	|H_{\text{PF}}(\omega)|^2=\frac{\displaystyle\frac{\omega^2}{\omega_c^2}}{\displaystyle1+7\frac{\omega^2}{\omega_c^2}+\frac{\omega^4}{\omega_c^4}};\\
	|H_{\text{RF}}(\omega)|^2=\frac{\displaystyle1+2\frac{\omega^2}{\omega_c^2}+\frac{\omega^4}{\omega_c^4}}{\displaystyle1+7\frac{\omega^2}{\omega_c^2}+\frac{\omega^4}{\omega_c^4}},
\end{gather}
possuem apenas potências pares de $\omega$ em sua representação. O que é esperado, dada a paridade simétrica da magnitude. Por outro lado, as respostas em fase
\begin{gather}
	\angle H_{\text{PB}}(\omega)=-\tan^{-1}\Big(\frac{\omega}{\omega_c}\Big);\\
	\angle H_{\text{PA}}(\omega)=\atan(\omega_c\omega,\omega^2);;\\
	\angle H_{\text{PF}}(\omega)=\atan\Big(\frac{\omega_c^2\omega-\omega^3}{3},\omega_c\omega^2\Big);\\
	\begin{aligned}
	\angle H_{\text{RF}}(\omega)=\atan\bigg(\frac{\omega_c\omega(\omega^2-\omega_c^2)}{\omega^4+7\omega_c^2\omega^2+\omega_c^4},\\\frac{\omega^4+4\omega_c^2\omega^2+\omega_c^4}{\omega^4+7\omega_c^2\omega^2+\omega_c^4}\bigg),\end{aligned}
\end{gather}
apresentam apenas potências ímpares\footnote{O uso da função $\atan(\cdot)$ mascara essa relação.} de $\omega$, dada a anti"-simetria de sua paridade.

As respostas em magnitude e em fase podem ser visualizadas nos diagramas de Bode das Figuras~\ref{fig:bodeHlp}--\ref{fig:bodeHsb}.

\begin{figure}[htbp]
	\centering
	\tikzsetnextfilename{bodeHlp}
	\begin{tikzpicture}
		\begin{semilogxaxis}[width=0.7\textwidth, xlabel=$\omega/\omega_c$, axis y line*=left, ylabel=$|H(\omega)|^2\slash\unit{\decibel}$, xmin=0.01, xmax=100, xmajorgrids=true]
			\addplot[thick] gnuplot [id=powHlp,raw gnuplot]{%
				set logscale x 10;
				f(x)=1/(1+x**2);
				plot [0.01:100] 10*log10(f(x))};
		\end{semilogxaxis}
		\begin{semilogxaxis}[width=0.7\textwidth, hide x axis, axis y line*=right, ylabel=\textcolor{gray}{$\angle H(\omega)\slash\pi\unit{\radian}$}, xmin=0.01, xmax=100, ytick={0,-0.25,-0.5}, yticklabels={$0$,$-\frac{1}{4}$,$-\frac{1}{2}$}]
			\addplot[gray, thick] gnuplot [id=argHlp,raw gnuplot]{%
				set logscale x 10;
				f(x)=-atan(x);
				plot [0.01:100] f(x)/pi};
		\end{semilogxaxis}
	\end{tikzpicture}
	\caption{Diagrama de Bode do filtro $H_{\text{PB}}(\omega)$.}
	\label{fig:bodeHlp}
\end{figure}

\begin{figure}[htbp]
	\centering
	\tikzsetnextfilename{bodeHhp}
	\begin{tikzpicture}
		\begin{semilogxaxis}[width=0.7\textwidth, xlabel=$\omega/\omega_c$, axis y line*=left, ylabel=$|H(\omega)|^2\slash\unit{\decibel}$, xmin=0.01, xmax=100, xmajorgrids=true]
			\addplot[thick] gnuplot [id=powHhp,raw gnuplot]{%
				set logscale x 10;
				f(x)=x**2/(1+x**2);
				plot [0.01:100] 10*log10(f(x))};
		\end{semilogxaxis}
		\begin{semilogxaxis}[width=0.7\textwidth, hide x axis, axis y line*=right, ylabel=\textcolor{gray}{$\angle H(\omega)\slash\pi\unit{\radian}$}, xmin=0.01, xmax=100, ytick={0,0.25,0.5}, yticklabels={$0$,$\frac{1}{4}$,$\frac{1}{2}$}]
			\addplot[gray, thick] gnuplot [id=argHhp,raw gnuplot]{%
				set logscale x 10;
				f(x)=atan2(x,x**2);
				plot [0.01:100] f(x)/pi};
		\end{semilogxaxis}
	\end{tikzpicture}
	\caption{Diagrama de Bode do filtro $H_{\text{PA}}(\omega)$.}
	\label{fig:bodeHhp}
\end{figure}

\begin{figure}[htbp]
	\centering
	\tikzsetnextfilename{bodeHbp}
	\begin{tikzpicture}
		\begin{semilogxaxis}[width=0.7\textwidth, xlabel=$\omega/\omega_c$, axis y line*=left, ylabel=$|H(\omega)|^2\slash\unit{\decibel}$, xmin=0.01, xmax=100, xmajorgrids=true]
			\addplot[thick] gnuplot [id=powHbp,raw gnuplot]{%
				set logscale x 10;
				f(x)=x**2/(1+7*x**2+x**4);
				plot [0.01:100] 10*log10(f(x))};
		\end{semilogxaxis}
		\begin{semilogxaxis}[width=0.7\textwidth, hide x axis, axis y line*=right, ylabel=\textcolor{gray}{$\angle H(\omega)\slash\pi\unit{\radian}$}, xmin=0.01, xmax=100, ytick={0.5,0,-0.5}, yticklabels={$\frac{1}{2}$,$0$,$-\frac{1}{2}$}]
			\addplot[gray, thick] gnuplot [id=argHbp,raw gnuplot]{%
				set logscale x 10;
				f(x)=atan2((x-x**3)/3,x**2);
				plot [0.01:100] f(x)/pi};
		\end{semilogxaxis}
	\end{tikzpicture}
	\caption{Diagrama de Bode do filtro $H_{\text{PF}}(\omega)$.}
	\label{fig:bodeHbp}
\end{figure}

\begin{figure}[htbp]
	\centering
	\tikzsetnextfilename{bodeHsb}
	\begin{tikzpicture}
		\begin{semilogxaxis}[width=0.7\textwidth, xlabel=$\omega/\omega_c$, axis y line*=left, ylabel=$|H(\omega)|^2\slash\unit{\decibel}$, xmin=0.01, xmax=100, xmajorgrids=true]
			\addplot[thick] gnuplot [id=powHsb,raw gnuplot]{%
				set logscale x 10;
				f(x)=(1+2*x**2+x**4)/(1+7*x**2+x**4);
				plot [0.01:100] 10*log10(f(x))};
		\end{semilogxaxis}
		\begin{semilogxaxis}[width=0.7\textwidth, hide x axis, axis y line*=right, ylabel=\textcolor{gray}{$\angle H(\omega)\slash\pi\unit{\radian}$}, xmin=0.01, xmax=100, ytick={0.0625,0,-0.0625}, yticklabels={$\frac{1}{16}$,$0$,$-\frac{1}{16}$}, scaled ticks=false]
			\addplot[gray, thick] gnuplot [id=argHsb,raw gnuplot]{%
				set logscale x 10;
				f(x)=atan2((x**3-x)/(x**4+7*x**2+1),(x**4+4*x**2+1)/(x**4+7*x**2+1));
				plot [0.01:100] f(x)/pi};
		\end{semilogxaxis}
	\end{tikzpicture}
	\caption{Diagrama de Bode do filtro $H_{\text{RF}}(\omega)$.}
	\label{fig:bodeHsb}
\end{figure}

A interpretação física da resposta em magnitude é \enquote{óbvia}. Ela representa o ganho (ou atenuação) do filtro para uma determinada frequência. Infelizmente, o mesmo não pode ser dito da resposta em fase. Sua interpretação não é tão imediata e o fato de ser dada em radianos não ajuda.

Ainda assim, a fase é um aspecto importantíssimo da resposta do filtro. Para tentar trazer um melhor entendimento, vamos apresentar duas representações alternativas: o \emph{atraso de fase}; e o \emph{atraso de grupo}.

\section{Atraso de fase}

O atraso de fase é definido por
\begin{equation}
	\tau_p(\omega)=-\frac{\angle H(\omega)}{\omega},
\end{equation}
sendo uma grandeza dada em segundos. Embora o atraso de fase seja relevante em algumas aplicações de telecomunicações, seus efeitos não são importantes para a análise de filtros seletivos em frequência. Esse tópico não será expandido.

A \fig{taup} ilustra o efeito do atraso de fase. Nele um pulso senoidal com envelope gaussiano sofre atraso de fase. Entre as curvas preta e cinza, é possível visualizar o deslocamento, porém este efeito não altera o envelope da função (curvas pontilhadas).

\begin{figure}[ht]
	\centering
	\tikzsetnextfilename{taup}
	\begin{tikzpicture}
		\begin{axis}[xlabel=$t$, xmin=0, xmax=6, grid=major]
			\addplot[gray, thick] gnuplot [id=taup1, raw gnuplot]{%
				set samples 801;
				f(x)=exp(-(x-3)**2)*sin(8*pi*x);
				plot [0:6] f(x)};
		\addplot[thick] gnuplot [id=taup2, raw gnuplot]{%
				set samples 801;
				f(x)=exp(-(x-3)**2)*cos(8*pi*x);
				plot [0:6] f(x)};
		\addplot[thick, dotted] gnuplot [id=taup3, raw gnuplot]{%
				f(x)=exp(-(x-3)**2);
				plot [0:6] f(x)};
		\addplot[thick, dotted] gnuplot [id=taup4, raw gnuplot]{%
				f(x)=-exp(-(x-3)**2);
				plot [0:6] f(x)};
		\end{axis}
	\end{tikzpicture}
	\caption{Representação do atraso de fase.}
	\label{fig:taup}
\end{figure}

\section{Atraso de grupo}

O atraso de fase é definido por
\begin{equation}
	\tau_g(\omega)=-\diff{\angle H(\omega)}{\omega},
\end{equation}
sendo uma grandeza dada em segundos. Sua interpretação é de grande valia na análise de filtros seletivos em frequência pois, a \emph{variação do atraso de grupo} indica uma \emph{distorção de fase} do sinal.

Considerando as respostas dos filtros analisados neste Capítulo, os atrasos de grupo podem ser dados por:
\begin{gather}
	H_{\text{PB}}(\omega)\therefore\tau_g(\omega)=\frac{\tau}{1+\tau^2\omega^2};\\
	H_{\text{PA}}(\omega)\therefore\tau_g(\omega)=\frac{\tau}{1+\tau^2\omega^2};\\
	H_{\text{PF}}(\omega)\therefore\tau_g(\omega)=\frac{3\tau(1+\tau^2\omega^2)}{1+7\tau^2\omega^2+\tau^4\omega^4};\\
	H_{\text{RF}}(\omega)\therefore\tau_g(\omega)=\frac{\tau-8\tau^3\omega^2+\tau^5\omega^4}{(1+7\tau^2\omega^2+\tau^4\omega^4)(1+\tau^2\omega^2)}.\raisetag{-1em}
\end{gather}

A \fig{taug} ilustra o efeito do atraso de grupo. Aqui percebemos o deslocamento do envelope (linhas pontilhada e tracejadas) entre os sinais representados em preto e cinza.

\begin{figure}[ht]
	\centering
	\tikzsetnextfilename{taug}
	\begin{tikzpicture}
		\begin{axis}[xlabel=$t$, xmin=0, xmax=8, grid=major]
			\addplot[gray, thick] gnuplot [id=taug1, raw gnuplot]{%
				set samples 901;
				f(x)=exp(-(x-5)**2)*cos(8*pi*x);
				plot [0:8] f(x)};
		\addplot[thick] gnuplot [id=taug2, raw gnuplot]{%
				set samples 901;
				f(x)=exp(-(x-3)**2)*cos(8*pi*x);
				plot [0:8] f(x)};
		\addplot[thick, dotted] gnuplot [id=taug3, raw gnuplot]{%
				f(x)=exp(-(x-3)**2);
				plot [0:8] f(x)};
		\addplot[thick, dotted] gnuplot [id=taug4, raw gnuplot]{%
				f(x)=-exp(-(x-3)**2);
				plot [0:8] f(x)};
		\addplot[thick, dashed] gnuplot [id=taug5, raw gnuplot]{%
				f(x)=exp(-(x-5)**2);
				plot [0:8] f(x)};
		\addplot[thick, dashed] gnuplot [id=taug6, raw gnuplot]{%
				f(x)=-exp(-(x-5)**2);
				plot [0:8] f(x)};
		\end{axis}
	\end{tikzpicture}
	\caption{Representação do atraso de grupo.}
	\label{fig:taug}
\end{figure}

A \fig{dist} ilustra graficamente os efeitos da distorção de fase pela variação do atraso de grupo. Nela, um sinal bitonal com envelope gaussiano teve a fase de um dos tons alterada. Um ponto notável distinto está no pulso central ($t\approx\num{2.5}$). No sinal original, à esquerda, temos um pico isolado e bem definido. Já no sinal distorcido, à direita, o pico parece ter se subdividido em dois picos menores (\eng{peak splitting}). Esse fenômeno é típico do mau processamento dos sinais de eletrocardiograma, especialmente no complexo QRS.

\begin{figure}[ht]
	\centering
	\tikzsetnextfilename{dist1}
	\begin{tikzpicture}
		\begin{axis}[width=0.35\textwidth, xlabel=$t$, xmin=0, xmax=6, grid=major]
			\addplot[thick] gnuplot [id=dist1, raw gnuplot]{%
				set samples 2001;
				f(x)=0.5*exp(-(x-3)**2)*(cos(4*pi*x)+cos(12*pi*x));
				plot [0:6] f(x)};
		\end{axis}
	\end{tikzpicture}
	\tikzsetnextfilename{dist2}
	\begin{tikzpicture}
		\begin{axis}[width=0.35\textwidth, xlabel=$t$, xmin=0, xmax=6, grid=major]
			\addplot[thick] gnuplot [id=dist2, raw gnuplot]{%
				set samples 2001;
				f(x)=0.5*exp(-(x-3)**2)*(cos(4*pi*x)+sin(12*pi*x));
				plot [0:6] f(x)};
		\end{axis}
	\end{tikzpicture}
	\caption{Ilustrando a distorção de fase.}
	\label{fig:dist}
\end{figure}

\section{Polos da função de transferência}

Uma função de tranferência realizável $H(s)$ é formada por: um constante multiplicativa, zeros\footnote{Raízes do polinômio no numerador.} e polos\footnote{Raízes do polinômio no denominador.}. Respeitando as condições de realizabilidade, o filtro precisa ser causal e estável: de maneira que seus polos \emph{devem} possuir parte real \emph{estritamente negativa}.

Há duas situações possíveis então: polos reais e pares de polos complexos conjugados.

\subsection{Polos reais}

Quando a raiz do denominador é real, o termo pode ser parametrizado por
 \begin{equation}
	 H(s)=\frac{1}{\tau s+1},
 \end{equation}
sendo $\tau$ a constante de tempo ($\tau>0$) e o polo $s=-1\slash\tau$ é estritamente negativo. Sistemas dessa natureza possuem resposta temporal no formato
 \begin{equation}
	 h(t)=Ae^{-t\slash\tau}\quad t\geq0,
 \end{equation}
onde $A$ é uma constante a determinar --- em conjunto com os outros polos do filtro e eventuais condições iniciais em um \ac{pvi}

Reparametrizando a constante de tempo como uma frequência de corte $\omega_c=1\slash\tau$. Escrevemos a resposta em frequência em termos da resposta em magnitude
\begin{equation}
	|H(\omega)|^2=\frac{1}{\displaystyle1+\frac{\omega^2}{\omega_c^2}}
\end{equation}
e do atraso de grupo
\begin{equation}
	\tau_g(\omega)=\frac{\tau}{\displaystyle1+\frac{\omega^2}{\omega_c^2}}.
\end{equation}

Considerando que a relação entre a frequência radial e a frequência ordinária (em \unit{\hertz}) é dada por $\omega=2\pi f$, vamos, mais uma vez, reparamterizar a resposta em magnitude e o atraso de grupo:
\begin{gather}
	|H(f)|^2=\frac{1}{\displaystyle1+\frac{f^2}{f_c^2}};\\
	\tau_g(f)=\frac{\tau}{\displaystyle1+\frac{f^2}{f_c^2}}.
\end{gather}

Finalmente, vamos expressar a resposta em magnitude em decibéis, pois $|H(f)|_{\unit{\decibel}}^2=10\log|H(f)|^2$, de forma que
\begin{equation}
	|H(f)|_{\unit{\decibel}}^2=-10\log\Big(1+\frac{f^2}{f_c^2}\Big).
\end{equation}

E, normalizando o atraso de grupo pela constante de tempo, podemos escrever
\begin{equation}
	\frac{\tau_g(f)}{\tau}=\frac{1}{\displaystyle1+\frac{f^2}{f_c^2}}.
\end{equation}

Com isso, podemos plotar o diagrama de Bode modificado da \fig{poloReal}. Onde a relação entre os parâmetros $\tau$ e $f_c$ é dada por
\begin{equation}
	f_c=\frac{1}{2\pi\tau}.
\end{equation}

\begin{figure}[ht]
	\centering
	\tikzsetnextfilename{poloReal}
	\begin{tikzpicture}
		\begin{semilogxaxis}[width=0.7\textwidth, xlabel=$f/f_c$, axis y line*=left, ylabel=$|H(f)|^2\slash\unit{\decibel}$, xmin=0.01, xmax=100, ymin=-40, ymax=0, ytick={0,-3.0103,-20,-40}, yticklabels={$0$,$-3$,$-20$,$-40$}, grid=major]
			\addplot[thick] gnuplot [id=powPoloReal,raw gnuplot]{%
				set logscale x 10;
				f(x)=-10*log10(1+x**2);
				plot [0.01:100] f(x)};
		\end{semilogxaxis}
		\begin{semilogxaxis}[width=0.7\textwidth, hide x axis, axis y line*=right, ylabel=\textcolor{gray}{$\tau_g(f)\slash\tau$}, xmin=0.01, xmax=100, ymin=0, ymax=1, ytick={0,0.5,1}]
			\addplot[gray, thick] gnuplot [id=gdPoloReal,raw gnuplot]{%
				set logscale x 10;
				f(x)=1/(1+x**2);
				plot [0.01:100] f(x)};
		\end{semilogxaxis}
	\end{tikzpicture}
	\caption{Diagrama de Bode (modificado) de um polo real.}
	\label{fig:poloReal}
\end{figure}

A análise da \fig{poloReal} evidencia o comportamento assintóticos da resposta em frequência, bem como o critério para a frequência de corte $f_c$ (ou $\omega_c$).
\begin{gather*}
	f\ll f_c\therefore\frac{f^2}{f_c^2}\to0\implies|H(f)|_{\unit{\decibel}}^2\approx-10\log(1)=\qty{0}{\decibel};\\
	f\ll f_c\therefore\frac{f^2}{f_c^2}\to0\implies\tau_g(f)\approx\tau;\\
	f=f_c\therefore\frac{f^2}{f_c^2}=1\implies|H(f)|_{\unit{\decibel}}^2=-10\log(\textstyle\frac{1}{2})\approx\qty{-3}{\decibel};\\
	f=f_c\therefore\frac{f^2}{f_c^2}=1\implies\tau_g(f)=\frac{\tau}{2};\\
	f\gg f_c\therefore\Big(1+\frac{f^2}{f_c^2}\Big)\to\frac{f^2}{f_c^2}\implies|H(f)|_{\unit{\decibel}}^2\approx-20\log\Big(\frac{f}{f_c}\Big);\\
	f\gg f_c\therefore\frac{f^2}{f_c^2}\to\infty\implies\tau_g(f)\approx0.
\end{gather*}

Essa análise pode ser compreendida da seguinte forma:
\begin{itemize}
	\item Para frequências muito menores que a frequência de corte, a resposta em magnitude é plana (\qty{0}{\decibel}) e o atraso de grupo é constante ($\tau$~\unit{\second});
	\item Na frequência de corte, o ganho de potência é exatamente \num{0.5}, o que equivale à \qty{-3.01}{\decibel}, aproximadamente. O ganho correspondente em amplitude é de $1\slash\sqrt{2}=\sqrt{2}\slash2$, aproximadamente \num{0.707}. Na vizinhança da frequência de corte ($\frac{f_c}{10}<f<10f_c$) também observamos quase a totalidade da variação do atraso de grupo --- distorção --- do filtro;
	\item Para frequências muito maiores que a frequência de corte, a resposta em magnitude é uma reta com inclinação de $-20$~\unit{dB}\slash dec (decibéis por década). O atraso de grupo se aproxima, assintoticamente, de zero.
\end{itemize}

\subsection{Par de polos complexos conjugados}

Quando a raiz do denominador é quadrática e irredutível em reais, é comum parametrizá"-la por
\begin{equation}
	H(s)=\frac{\omega_0^2}{\displaystyle s^2+\frac{\omega_0}{Q}s+\omega_0^2},
\end{equation}
sendo $\omega_0$ a \emph{frequência natural não"-amortecida} ($\omega_0>0$ em \unit{\radian\per\second}) e $Q$ é o fator de qualidade ($Q>0$ adimensional). A presença de $\omega_0^2$ no numerado apenas normaliza a assíntota da resposta em frequência, como será visto a seguir.

Com um denominador quadrático, há três tipos possíveis de resposta, de acordo com o discriminante $\Delta$ da \emph{solução quadrática}\footnote{Conhecida no Brasil, incorretamente, como fórmula de Bhaskara.}
\begin{equation}
	\Delta=\frac{\omega_0^2}{Q^2}-4\omega_0^2,
\end{equation}
de maneira que:
\begin{itemize}
	\item $\Delta>0\implies Q<\frac{1}{2}$ --- duas raízes reais e distintas;
	\item $\Delta=0\implies Q=\frac{1}{2}$ --- duas raízes reais e iguais;
	\item $\Delta<0\implies Q>\frac{1}{2}$ --- um par de raízes complexas conjugadas.
\end{itemize}

Ambas as soluções com raízes reais não são relevantes ao problema de filtragem analógica --- exceto pelo já exposto na Subseção anterior --- e não serão consideradas aqui. A solução com um par de raízes complexas conjugadas, por outro lado, é de grande relevância no projeto de filtros analógicos (e digitais). Tais sistemas são chamados \emph{subamortecidos}.

A resposta temporal de um sistema subamortecido é dada, de uma forma geral, por
\begin{equation}
	h(t)=Ae^{-\alpha t}\cos(\beta t+\phi)\quad t\geq0,
\end{equation}
com $A$ e $\phi$ constantes a determinar pelo \ac{pvi}, enquanto os parâmetros $\alpha$ e $\beta$ são dados por:
\begin{gather}
	\alpha=\frac{\omega_0}{2Q},\\
	\beta=\frac{\sqrt{4Q^2-1}}{2Q}\omega_0.
\end{gather}

Aqui nota"-se uma confusão muito comum: a frequência de oscilação $\beta$ depende de $\omega_0$ mas é diferente dela. A frequência $\beta$ é chamada de \emph{frequência natural amortecida}. É simples averiguar que se $Q\to\infty$ então $\beta\to\omega_0$. Porém o fator de qualidade é limitado por aspectos tecnológicos da construção de circuitos e de componentes realizáveis.

A magnitude quadrática do sistema subamortecido é dada por
\begin{equation}
	|H(\omega)|^2=\frac{\omega_0^4}{\displaystyle\omega^4+\frac{\omega_0^2-2Q^2\omega_0^2}{Q^2}\omega^2+\omega_0^4},
\end{equation}
e o atraso de grupo por
\begin{equation}
	\tau_g(\omega)=\frac{\displaystyle\frac{\omega_0}{Q}(\omega^2+\omega_0^2)}{\displaystyle\omega^4+\frac{\omega_0^2-2Q^2\omega_0^2}{Q^2}\omega^2+\omega_0^4}.
\end{equation}

Aqui, a substituição $\omega=2\pi f$ não beneficia a análise, de maneira que vamos continuar com $\omega$. A análise dos pontos estacionários (derivada nula) de $|H(\omega)|^2$ revela um novo ponto notável: a \emph{frequência de ressonância} $\omega_r$.

A frequência
\begin{equation}
	\omega_r=\frac{\omega_0\sqrt{4Q^2-2}}{2Q},
\end{equation}
indica um máximo da resposta em magnitude, desde que a condição
\begin{equation}
	Q\geq\frac{\sqrt{2}}{2},
\end{equation}
seja satisfeita --- lembrando que já assumimos que $Q>\frac{1}{2}$.

Assim, ressalto aqui a potencial confusão de \enquote{frequências} quando lidamos com sistemas subamortecidos: temos a frequência natural não"-amortecida $\omega_0$; a frequência natural amortecida $\beta$ --- observada na resposta temporal; e a frequência de ressonância $\omega_r$ que é um ponto notável no diagrama de Bode. Tanto $beta$ quanto $\omega_r$ convergem para $\omega_0$ se $Q\to\infty$.

O gráfico da \fig{poloConjMag} traz uma pequena coleção de curvas no diagrama de Bode para a resposta em magnitude de um sistema subamortecido com $Q=\{1,2,10\}$. Nele observamos o pico de ressonância na resposta em magnitude, cada vez mais pronunciado, conforme $Q$ aumenta. O comportamento assintótico para $\omega\to0$ é de assentar em \qty{0}{\decibel}, enquanto para $\omega\to\infty$ temos uma reta de \qty{-40}{\decibel\slash dec}

\begin{figure}[ht]
	\centering
	\tikzsetnextfilename{poloConjMag}
	\begin{tikzpicture}
		\begin{semilogxaxis}[xlabel=$\omega/\omega_0$, ylabel=$\omega/\omega_0$, ylabel=$|H(\omega)|^2\slash\unit{\decibel}$, xmin=0.1, xmax=10, grid=major]
			\addplot[thick, dotted] gnuplot [id=powPoloConjMagQ1, raw gnuplot]{%
				set logscale x 10;
				Q=1.0;
				g(x)=x**4+(1-2*Q**2)/(Q**2)*x**2+1;
				f(x)=-10*log10(g(x));
				plot [0.1:10] f(x)};
			\addplot[thick, dashed] gnuplot [id=powPoloConjMagQ2, raw gnuplot]{%
				set logscale x 10;
				Q=2.0;
				g(x)=x**4+(1-2*Q**2)/(Q**2)*x**2+1;
				f(x)=-10*log10(g(x));
				plot [0.1:10] f(x)};
			\addplot[thick] gnuplot [id=powPoloConjMagQ3, raw gnuplot]{%
				set samples 301;
				set logscale x 10;
				Q=10.0;
				g(x)=x**4+(1-2*Q**2)/(Q**2)*x**2+1;
				f(x)=-10*log10(g(x));
				plot [0.1:10] f(x)};
			\legend{$Q=1$, $Q=2$, $Q=10$};
		\end{semilogxaxis}
	\end{tikzpicture}
	\caption{Diagrama de Bode (magnitude) de um sistema subamortecido.}
	\label{fig:poloConjMag}
\end{figure}

Já na \fig{poloConjTaug} representa uma coleção de atrasos de grupo para o sistema subamortecido, também para $Q=\{1,2,10\}$. Aqui vemos que a variação do atraso de grupo possui comportamento similar ao pico de ressonância. Lembramos que a variação de atraso de grupo se traduz como distorção do sinal pelo filtro.

\begin{figure}[ht]
	\centering
	\tikzsetnextfilename{poloConjTaug}
	\begin{tikzpicture}
		\begin{semilogxaxis}[xlabel=$\omega/\omega_0$, ylabel=$\tau_g(\omega)\slash\unit{\second}$, xmin=0.1, xmax=10, ymin=0, grid=major]
			\addplot[thick, dotted] gnuplot [id=powPoloConjTaugQ1, raw gnuplot]{%
				set logscale x 10;
				Q=1.0;
				f(x)=(x**2+1)/Q;
				g(x)=x**4+(1-2*Q**2)/(Q**2)*x**2+1;
				plot [0.1:10] f(x)/g(x)};
			\addplot[thick, dashed] gnuplot [id=powPoloConjTaugQ2, raw gnuplot]{%
				set samples 200;
				set logscale x 10;
				Q=2.0;
				f(x)=(x**2+1)/Q;
				g(x)=x**4+(1-2*Q**2)/(Q**2)*x**2+1;
				plot [0.1:10] f(x)/g(x)};
			\addplot[thick] gnuplot [id=powPoloConjTaugQ3, raw gnuplot]{%
				set samples 999;
				set logscale x 10;
				Q=10.0;
				f(x)=(x**2+1)/Q;
				g(x)=x**4+(1-2*Q**2)/(Q**2)*x**2+1;
				plot [0.1:10] f(x)/g(x)};
			\legend{$Q=1$, $Q=2$, $Q=10$};
		\end{semilogxaxis}
	\end{tikzpicture}
	\caption{Diagrama de Bode (atraso de grupo) de um sistema subamortecido.}
	\label{fig:poloConjTaug}
\end{figure}
