\chapter{Aproximações reais}

No Capítulo anterior, foi analisado em detalhe a resposta em frequência de circuitos com polos reais ou pares de polos complexos conjugados. A literatura especializada oferece grande abundância de circuitos que implementam funções de transferência dessa natureza, porém, isso nem sempre é suficiente para atender às especificações de um problema de filtragem.

Neste Capítulo, discutiremos a questão da especificação de uma resposta em frequência, bem como a ideia das aproximações reais do problema de filtragem analógica seletiva em frequência.

Para simplicidade do desenvolvimento matemático, as aproximações serão vistas tomando um \emph{protótipo passa"-baixas normalizado}, ou seja, um filtro passa"-baixas cuja frequência de corte é de \qty{1}{\radian\per\second} ($\approx$\qty{0.16}{\hertz}).

Adicionalmente, por simplicidade da notação, vamos definir uma grandeza chamada de \emph{atenuação}, denotada por $A(\omega)$. Sua relação com a magnitude quadrática é dada por
\begin{equation}
	A(\omega)=\frac{1}{|H(\omega)|^2}.
\end{equation}

A seguir serão discutidas e apresentadas as particularidades das aproximações de Butterworth, Chebyshev e Bessel. Essas aproximações possuem equacionamento mais simples, pois seus filtros passa"-baixa possuem apenas polos (\eng{all-poles}).

\section{Aproximação de Butterworth}

Em 1930 o físico inglês Stephen Butterworth publicou um arigo intulado \eng{\enquote{On the Theory of Filter Amplifiers}}. Neste trabalho, Butterworth defendeu a ideia que bons filtros não apresentam oscilação na resposta em magnitude, ou seja, possuem resposta monotônica. Um filtro passa"-baixas de Butterworth possui resposta em magnitude estritamente decrescente em função da frequência.

A solução apresentada por Butterworth para um protótipo passa"-baixas normalizado possui a seguinte resposta em magnitude:
\begin{equation}\label{eq:buttap}
	|H(\omega)|^2=\frac{1}{1+\omega^{2N}},
\end{equation}
onde $N\in\mathbb{N}^*$ é a ordem\footnote{Equivalente ao número de componentes reativos irredutíveis por associação.} do filtro.

A atenuação do filtro é dada então por $A(\omega)=1+\omega^{2N}$. A inspeção da \fig{abuttap} indica que não há oscilações na resposta em magnitude --- a atenuação é uma função crescente de $\omega$. Independente da ordem $N$, a atenuação sempre vale \qty{3}{\decibel} para $\omega=$\qty{1}{\radian\per\second}.

\begin{figure}[ht]
	\centering
	\tikzsetnextfilename{abuttap}
	\begin{tikzpicture}
		\begin{semilogxaxis}[xlabel=$\omega$\slash(\unit{\radian\per\second}), ylabel=$A(\omega)$\slash\unit{\decibel}, xmin=0.01, xmax=100, ymin=0, ymax=60, grid=major, legend pos=north west]
			\addplot[thick, dotted] gnuplot [id=abuttap1, raw gnuplot]{%
				set logscale x 10;
				N = 1;
				f(x)=1+x**(2*N);
				plot [0.01:100] 10*log10(f(x))};
			\addplot[thick, dashdotted] gnuplot [id=abuttap2, raw gnuplot]{%
				set logscale x 10;
				N = 2;
				f(x)=1+x**(2*N);
				plot [0.01:100] 10*log10(f(x))};
			\addplot[thick, dashed] gnuplot [id=abuttap3, raw gnuplot]{%
				set logscale x 10;
				N = 3;
				f(x)=1+x**(2*N);
				plot [0.01:100] 10*log10(f(x))};
			\addplot[thick] gnuplot [id=abuttap4, raw gnuplot]{%
				set logscale x 10;
				N = 4;
				f(x)=1+x**(2*N);
				plot [0.01:100] 10*log10(f(x))};
			\legend{$N=1$, $N=2$, $N=3$, $N=4$};
		\end{semilogxaxis}
	\end{tikzpicture}
	\caption{Atenuação do filtro de Butterworth.}
	\label{fig:abuttap}
\end{figure}

O filtro de Butterworth é conhecido por ter \emph{banda de passagem maximamente plana}. Isso significa que para frequências abaixo da frequência de corte, o ganho é tão próximo da unidade (\qty{0}{\decibel}) quanto possível. A análise das derivadas de $A(\omega)$ na origem indicam isso:
\begin{align*}
	\diffn{A(\omega)}{\omega}{0}=1+\omega^{2N}&\implies\diffn{A(\omega)}{\omega}{0}\bigg|_{\omega=0}=1;\\
	\diffn{A(\omega)}{\omega}{1}=2N\omega^{2N-1}&\implies\diffn{A(\omega)}{\omega}{1}\bigg|_{\omega=0}=0;\\
	\diffn{A(\omega)}{\omega}{2}=2N(2N-1)\omega^{2N-2}&\implies\diffn{A(\omega)}{\omega}{2}\bigg|_{\omega=0}=0;\\
	&\cdots\\
	\diffn{A(\omega)}{\omega}{2N}=(2N)!&\implies\diffn{A(\omega)}{\omega}{2N}\bigg|_{\omega=0}=(2N)!.\\
\end{align*}

Na origem, $A(\omega)$ é unitária, porém todas as suas derivadas são nulas, exceto a de mais alta ordem ($2N$). Isso indica uma forte tendência em manter o ganho unitário.

\subsection{Os polos do filtro de Butterworth}

A aproximação de Butterworth parece estar embasada em boas ideias. Resta a questão: \enquote{qual a função de transferência que possui uma resposta desse tipo?} Vamos discutir aqui, em detalhe, como são calculados os polos de Butterworth.

Conforme visto anteriormente, a resposta em frequência $H(\omega)$ é uma função complexa sobre a variável real $\omega$. Assim, sua magnitude quadrática pode ser escrita como $|H(\omega)|^2=H(\omega)\conj{H(\omega)}$, onde $\conj{(\cdot)}$ denota a conjugação complexa.

Em razão das propriedades de simetria conjugada da \ac{tf}, chegamos à equivalência $\conj{H(\omega)}=H(-\omega)$. Com isso, chegamos à relação
\begin{align*}
	|H(\omega)|^2&=H(\omega)\conj{H(\omega)};\\
	&=H(\omega)H(-\omega);\\
	&=H(s)H(-s)\Big|_{s=j\omega}.
\end{align*}

Assim, da mesma maneira que fazemos o mapeamento $s=j\omega$, podemos fazer o mapeamento inverso $\omega=-js$. Reescrevendo \equ{buttap}:
\begin{align}
	H(s)H(-s)&=\frac{1}{1+(-js)^{2N}};\notag\\
			 &=\frac{1}{1+(-1)^{2N}j^{2N}s^{2N}};\notag\\
			 &=\frac{1}{1+(-1)^Ns^{2N}}.
\end{align}

Os polos de $H(s)H(-s)$ são as raízes de $1+(-1)^Ns^{2N}$, que podem ser escritos por:
\begin{align}
	1+(-1)^Ns^{2N}&=0;\notag\\
	(-1)^Ns^{2N}&=-1;\notag\\
	(-1)^{2N}s^{2N}&=-1(-1)^N;\notag\\
	s^{2N}&=(-1)^{N+1}.
\end{align}

Assim, caso $N$ seja ímpar, buscamos as raízes de $s^{2N}=1$. Caso $N$ seja par, buscamos as raízes de $s^{2N}=-1$. Tomando as raízes de $s$ como $p$ (já que são polos do sistema) e aplicando o teorema de De Moivre, vamos analisar esses polos para $N=\{1,2,3,4\}$ e averiguar se há algum \eng{insight} possível.

Tomando $N=1$,
\begin{align}
	p^2&=(-1)^{2};\notag\\
	p^2&=1;\notag\\
	p_n^2&=\exp(j2\pi n);\\
	p_n&=\exp(j\pi n).
\end{align}

Tomando agora $N=2$,
\begin{align}
	p^4&=(-1)^{3};\notag\\
	p^4&=-1;\notag\\
	p^4&=\exp\big(j(\pi+2\pi n)\big);\notag\\
	p_n&=\exp\Big(j\frac{\pi}{4}+j\frac{\pi}{2}n\Big).
\end{align}

Tomando então $N=3$,
\begin{align}
	p^6&=(-1)^{4};\notag\\
	p^6&=1;\notag\\
	p^6&=\exp\big(j2\pi n\big);\notag\\
	p_n&=\exp\Big(j\frac{\pi}{3}n\Big).
\end{align}

Tomando finalmente $N=4$,
\begin{align}
	p^8&=(-1)^{5};\notag\\
	p^8&=-1;\notag\\
	p^8&=\exp\big(j(\pi+2\pi n)\big);\notag\\
	p_n&=\exp\Big(j\frac{\pi}{8}+j\frac{\pi}{4}n\Big).
\end{align}

A \fig{splane} ilustra as posições dos polos de $H(s)H(-s)$ para $N=\{1,2,3,4\}$. Nela, percebemos polos simétricos em relação à origem do plano $s$. Sem perda de generalidade, vamos considerar que $H(s)$ contém os polos causais e estáveis à esquerda do eixo $j\omega$, enquanto $H(-s)$ contém os polos não"-causais e estáveis (ou causais e instáveis) à direita do eixo $j\omega$.

\begin{figure}[hbtp]
\centering
\tikzsetnextfilename{splane}
\begin{tikzpicture}
\begin{axis}[title={$N=1$}, name=axis1, width=0.4\textwidth,height=0.4\textwidth, axis equal, axis x line=middle, axis y line=center, xlabel style={anchor=north}, xlabel={$\sigma$}, ylabel={$j\omega$}, xmin=-1.25, xmax=1.25, ymin=-1.25, ymax=1.25, xtick={-1,1}, ticklabel style={anchor=north east}, ytick={-1,1}]
    \addplot[dashed] gnuplot[id=pcirc, raw gnuplot]{
	f(x) = sqrt(1-x**2);
	plot[-1:1] f(x)};
	\addplot[dashed] gnuplot[id=ncirc, raw gnuplot]{
	f(x) = -sqrt(1-x**2);
	plot[-1:1] f(x)};
	\addplot[ultra thick, only marks, mark=x, mark size=6pt] coordinates {(-1,0) (1,0)};
\end{axis}
\begin{axis}[title={$N=2$}, name=axis2, at={($(axis1.east)+(1cm,0)$)}, anchor=west, width=0.4\textwidth,height=0.4\textwidth, axis equal, axis x line=middle, axis y line=center, xlabel style={anchor=north}, xlabel={$\sigma$}, ylabel={$j\omega$}, xmin=-1.25, xmax=1.25, ymin=-1.25, ymax=1.25, xtick={-1,1}, ticklabel style={anchor=north east}, ytick={-1,1}]
    \addplot[dashed] gnuplot[id=pcirc, raw gnuplot]{
	f(x) = sqrt(1-x**2);
	plot[-1:1] f(x)};
	\addplot[dashed] gnuplot[id=ncirc, raw gnuplot]{
	f(x) = -sqrt(1-x**2);
	plot[-1:1] f(x)};
	\addplot[ultra thick, only marks, mark=x, mark size=6pt] coordinates {(0.707,0.707) (-0.707,0.707) (-0.707,-0.707) (0.707,-0.707)};
\end{axis}
\begin{axis}[title={$N=3$}, name=axis3, at={($(axis1.south)-(0,1cm)$)}, anchor=north, width=0.4\textwidth,height=0.4\textwidth, axis equal, axis x line=middle, axis y line=center, xlabel style={anchor=north}, xlabel={$\sigma$}, ylabel={$j\omega$}, xmin=-1.25, xmax=1.25, ymin=-1.25, ymax=1.25, xtick={-1,1}, ticklabel style={anchor=north east}, ytick={-1,1}]
    \addplot[dashed] gnuplot[id=pcirc, raw gnuplot]{
	f(x) = sqrt(1-x**2);
	plot[-1:1] f(x)};
	\addplot[dashed] gnuplot[id=ncirc, raw gnuplot]{
	f(x) = -sqrt(1-x**2);
	plot[-1:1] f(x)};
	\addplot[ultra thick, only marks, mark=x, mark size=6pt] coordinates {(1,0) (0.5,0.866) (-0.5,0.866) (-1,0) (-0.5,-0.866) (0.5,-0.866)};
\end{axis}
\begin{axis}[title={$N=4$}, name=axis4, at={($(axis3.east)+(1cm,0)$)}, anchor=west, width=0.4\textwidth,height=0.4\textwidth, axis equal, axis x line=middle, axis y line=center, xlabel style={anchor=north}, xlabel={$\sigma$}, ylabel={$j\omega$}, xmin=-1.25, xmax=1.25, ymin=-1.25, ymax=1.25, xtick={-1,1}, ticklabel style={anchor=north east}, ytick={-1,1}]
    \addplot[dashed] gnuplot[id=pcirc, raw gnuplot]{
	f(x) = sqrt(1-x**2);
	plot[-1:1] f(x)};
	\addplot[dashed] gnuplot[id=ncirc, raw gnuplot]{
	f(x) = -sqrt(1-x**2);
	plot[-1:1] f(x)};
	\addplot[ultra thick, only marks, mark=x, mark size=6pt] coordinates {(0.924,0.383) (0.383,0.924) (-0.924,0.383) (-0.383,0.924) (-0.924,-0.383) (-0.383,-0.924) (0.924,-0.383) (0.383,-0.924)};
\end{axis}
\end{tikzpicture}
\caption{Polos de $H(s)H(-s)$, para a aproximação de Butterworth, no plano $s$.}
\label{fig:splane}
\end{figure}

Percebemos que os polos causais estão no semiplano esquerdo, com uma fase inicial de $\frac{pi}{2}+\frac{\pi}{2N}$, separados entre si por $\frac{\pi}{N}$. Com isso, podemos chegar às seguintes relações para o filtro de Butterworth de ordem $N$:
\begin{equation}\label{eq:buttpoles}
	\boxed{p_n=\exp\Big(j\frac{\pi}{2}-j\frac{\pi}{2N}+j\frac{\pi}{N}n\Big)\text{ para }n=1,\ldots,N,}
\end{equation}
com a função de transferência dada por
\begin{equation}\label{eq:tfbutt}
	\boxed{H(s)=\frac{1}{\displaystyle\prod_{n=1}^{N}(s-p_n)}.}
\end{equation}

No Matlab\textsuperscript{®}, o Signal Processing Toolbox implementa a \equ{buttpoles} através da função \lstinline{buttap()}.

\subsection{Os fatores de qualidade de Butterworth}

A análise das Eqs.~\eqref{eq:buttpoles} e \eqref{eq:tfbutt}, auxiliada pela \fig{splane}, nos permite concluir o seguinte:
\begin{itemize}
	\item o fator $(s+1)$, correspondente ao polo real, figura apenas para $N$ ímpar;
	\item os polos complexos figuram sempre em pares conjugados --- $(s-p_n)(s-\conj{p_n})=(s^2-2\Re(p_n)s+|p_n|^2)$.
\end{itemize}

O termo $-2\Re(p_n)$ pode ser escrito da forma
\begin{align}
	-2\Re(p_n)&=-2\Re\bigg(\exp\Big(j\frac{\pi}{2}-j\frac{\pi}{2N}+j\frac{\pi}{N}n\Big)\bigg)\notag\\
			  &=2\Re\bigg(\exp\Big(-j\frac{\pi}{2}-j\frac{\pi}{2N}+j\frac{\pi}{N}n\Big)\bigg)\notag\\
			  &=2\cos\Big(-\frac{\pi}{2}-\frac{\pi}{2N}+\frac{\pi}{N}n\Big)\notag\\
			  &=2\cos\Big(\frac{2\pi-1}{2N}\pi-\frac{\pi}{2}\Big)\notag\\
			  &=2\sin\Big(\frac{2\pi-1}{2N}\pi\Big).
\end{align}

Isso nos permite reescrever as \equ{tfbutt} como:
\begin{gather}
	\boxed{H(s)=\frac{1}{\displaystyle(s+1)\prod_{n=1}^{\frac{N-1}{2}}\bigg(s^2+2\sin\Big(\frac{2n-1}{2N}\pi\Big)s+1\bigg)}\text{ para }N\text{ ímpar};}\\
	\boxed{H(s)=\frac{1}{\displaystyle\prod_{n=1}^{\frac{N}{2}}\bigg(s^2+2\sin\Big(\frac{2n-1}{2N}\pi\Big)s+1\bigg)}\text{ para }N\text{ par}.}
\end{gather}

Lembrando a parametrização para pares de polos complexos conjugados: $s^2+\frac{\omega_0}{Q}s+\omega_0^2$. Podemos averiguar que para um protótipo passa"-baixas de Butterworth normalizado, $\omega_0=1$~\unit{\radian\per\second}.

Assim, exemplificando para $N=\{1,2,3,4\}$, temos, com $N=1$
\begin{equation}
	H(s)=\frac{1}{s+1},
\end{equation}
para $N=2$
\begin{equation}
	H(s)=\frac{1}{s^2+2\sin\big(\frac{\pi}{4}\big)s+1}=\frac{1}{s^2+\sqrt{2}s+1},
\end{equation}
já para $N=3$
\begin{equation}
	H(s)=\frac{1}{(s+1)\Big(s^2+2\sin\big(\frac{\pi}{6}\big)s+1\Big)}=\frac{1}{(s+1)(s^2+s+1)},
\end{equation}
e, finalmente, para $N=4$
\begin{align}
	H(s)&=\frac{1}{\Big(s^2+2\sin\big(\frac{\pi}{8}\big)s+1\Big)\Big(s^2+2\sin\big(\frac{3}{8}\pi\big)s+1\Big)}\notag\\
		&=\frac{1}{\Big(s^2+\sqrt{2-\sqrt{2}}s+1\Big)\Big(s^2+\sqrt{2+\sqrt{2}}s+1\Big)}.
\end{align}

Notamos que, para $N=4$, o máximo valor de $Q$ para um filtro de Butterworth passa"-baixas é de, aproximadamente, \num{1.31}.

\section{Aproximação de Chebyshev}

A aproximação de Chebyshev é nomeada em homenagem a matemático russo do século XIX, Pafnuty Lvovich Chebyshev\footnote{Não há consenso geral sobre a transliteração do nome em cirílico. Historicamente, as seguintes versões foram circuladas: Tchebichef, Tchebychev, Tchebycheff, Tschebyschev, Tschebyschef, Tschebyscheff, Čebyčev, Čebyšev, Čebyšëv, Chebysheff, Chebychov e Chebyshov. Essa última, considerada mais apropriada por nativos da língua russa. A transliteração aqui usada é a adotada pela American Mathematical Society (AMS).}. Registros históricos incompletos remetem sua proposição original ao matemático alemão Wilhelm Cauer (1900--1945). Seu livro \emph{Siebschaltungen}, publicado em 1931, faz a primeira menção do uso dos polinômios de Chebyshev na síntese de filtros seletivos em frequência. A maior parte dos trabalhos de Cauer foi perdida durante a guerra.

A magnitude quadrática de um protótipo passa"-baixas normalizado de Chebyshev é dada por
\begin{equation}
	|H(\omega)|^2=\frac{1}{1+\varepsilon^2T_N^2(\omega)},
\end{equation}
onde $\varepsilon$ é um parâmetro de tolerância na banda de passagem e $T_N(\omega)$ é o polinômio de Chebyshev de primeira espécie, definido por
\begin{equation}\label{eq:chebpol}
	T_N(\omega)=\begin{cases}
		\cos\big(N\cos^{-1}(\omega)\big)\quad&\text{se }|\omega|\leq1;\\
		\cosh\big(N\cosh^{-1}(\omega)\big)\quad&\text{se }|\omega|>1.
	\end{cases}
\end{equation}

Os polinômios de Chebyshev de primeira espécie obedecem à seguinte lei de recursão
\begin{equation}
	T_N(\omega)=2\omega T_{N-1}(\omega)-T_{N-2}(\omega),
\end{equation}
com $T_0(\omega)=1$ e $T_1(\omega)=\omega$.

A \equ{chebpol} evidencia que $T_N(\omega)$ varia entre $1$ e $-1$ para $\omega<1$ (ou seja, na banda de passagem). Assim, $T_N^2(\omega)$ varia entre $0$ e $1$ na banda de passagem. Considerando a atenuação do filtro de Chebyshev
\begin{equation}
	A(\omega)=1+\varepsilon^2T_N^2(\omega),
\end{equation}
a \fig{achebap} representa sua atenuação para $N=\{1,2,3,4\}$, com \eng{ripple} de banda de passagem $R_p$ (em \unit{\decibel}), dado por
\begin{equation}
	R_p=10\log(1+\varepsilon^2).
\end{equation}

\begin{figure}[ht]
	\centering
	\tikzsetnextfilename{achebap}
	\begin{tikzpicture}
		\begin{semilogxaxis}[xlabel=$\omega$\slash(\unit{\radian\per\second}), ylabel=$A(\omega)$\slash\unit{\decibel}, xmin=0.01, xmax=100, ymin=0, ymax=40, ytick={0,3.01,10,20,30,40}, yticklabels={0,$R_p$,10,20,30,40}, grid=major, legend pos=north west]
			\addplot[thick, dotted] gnuplot [id=achebap1, raw gnuplot]{%
				set logscale x 10;
				N = 1;
				epsilon = 1;
				f(x)=1+epsilon**2*cosh(N*acosh(x))**2;
				plot [0.01:100] 10*log10(f(x))};
			\addplot[thick, dashdotted] gnuplot [id=achebap2, raw gnuplot]{%
				set logscale x 10;
				N = 2;
				epsilon = 1;
				f(x)=1+epsilon**2*cosh(N*acosh(x))**2;
				plot [0.01:7] 10*log10(f(x))};
			\addplot[thick, dashed] gnuplot [id=achebap3, raw gnuplot]{%
				set logscale x 10;
				N = 3;
				epsilon = 1;
				f(x)=1+epsilon**2*cosh(N*acosh(x))**2;
				plot [0.01:3] 10*log10(f(x))};
			\addplot[thick] gnuplot [id=achebap4, raw gnuplot]{%
				set logscale x 10;
				N = 4;
				epsilon = 1;
				f(x)=1+epsilon**2*cosh(N*acosh(x))**2;
				plot [0.01:2] 10*log10(f(x))};
			\legend{$N=1$, $N=2$, $N=3$, $N=4$};
		\end{semilogxaxis}
	\end{tikzpicture}
	\caption{Atenuação do filtro de Chebyshev --- $R_p$ exagerado para visualização da oscilação em banda de passagem.}
	\label{fig:achebap}
\end{figure}

A apresentação da obtenção dos polos do filtro de Chebyshev é razoavelmente mais envolvida que a de Butterworth. Por essa razão, ela não será apresentada em detalhe. Para nossos propósitos, o Signal Processing Toolbox oferece a função \lstinline{cheb1ap()} que faz o cálculo dos polos.

\section{Aproximação de Bessel}

Nomeado em honra ao matemático alemão Friedrich Wilhelm Bessel (1784--1846). Dentre seus pioneiros, destacam"-se Z.\ Kiyasu e W.E.\ Thomson que, em 1943 e 1949, respectivamente, aplicaram as funções de Bessel ao projeto de filtros seletivos em frequência.

Diferentemente das aproximações de Butterworth e Chebyshev, que são definidas em termos da resposta em magntiude, as resposta de um filtro de Bessel é descrita por sua característica de fase. Particularmente, a aproximação de Bessel provê um atraso de grupo \emph{maximamente plano} na banda de passagem.

Ou seja, o desenvolvimento do filtro de Bessel segue linhas similares às do de Butterworth, porém, substituindo as características da magnitude quadrática pelos do atraso de grupo. Os pioneiros Kiyasu e Thomson foram os primeiros a perceber que os polinômios de Bessel reuniam essas características.

A função de transferência do filtro de Bessel é dada por
\begin{equation}
	H(s)=\frac{B_N(0)}{B_N(s)}=\frac{b_0}{\displaystyle\sum_{n=0}^Nb_ns^n},
\end{equation}
sendo
\begin{equation}
	b_n=\frac{(2N-n)!}{2^{N-n}n!(N-n)!}
\end{equation}
e $B_N(s)$ é o polinômio de Bessel de ordem $N$. Os polinomios de Bessel obedecem à recursão
\begin{equation}
	B_N(s)=(2N-1)B_{N-1}+s^2B_{N-2}(s),
\end{equation}
com $B_0(s)=1$ e $B_1(s)=s+1$.

No Signal Processing Toolbox, a função \lstinline{besselap()}\footnote{Os polos produzidos por \lstinline{besselap()} são escalados para que o comportamento assintótico do filtro de Bessel se sobreponha ao de Butterworth.} calcula os polos do filtro de Bessel. A \fig{taugbessel} ilustra os atrasos de grupo para $N=\{1,2,3,4\}$ para um protótipo passa"-baixas normalizado de Bessel.

\begin{figure}[ht]
	\centering
	\tikzsetnextfilename{taugbessel}
	\begin{tikzpicture}
		\begin{semilogxaxis}[xlabel=$\omega$\slash(\unit{\radian\per\second}), ylabel=$\tau_g(\omega)$\slash\unit{\second}, xmin=0.01, xmax=100, grid=major]
			\addplot[thick, dotted] gnuplot [raw gnuplot]{%
				set logscale x 10;
				plot 'data/taugbessel.dat' using 1:2};
			\addplot[thick, dashdotted] gnuplot [raw gnuplot]{%
				set logscale x 10;
				plot 'data/taugbessel.dat' using 1:3};
			\addplot[thick, dashed] gnuplot [raw gnuplot]{%
				set logscale x 10;
				plot 'data/taugbessel.dat' using 1:4};
			\addplot[thick] gnuplot [raw gnuplot]{%
				set logscale x 10;
				plot 'data/taugbessel.dat' using 1:5};
			\legend{$N=1$, $N=2$, $N=3$, $N=4$};
		\end{semilogxaxis}
	\end{tikzpicture}
	\caption{Atraso de grupo do protótipo passa"-baixas normalizado de Bessel.}
	\label{fig:taugbessel}
\end{figure}

A análise da \fig{taugbessel} nos informa que, para um filtro de Bessel, a \enquote{planaridade} do atraso de grupo melhora com o aumento da ordem. Essa é uma característica distinta deste filtro. Como veremos mais adiante, as aproximações de Butterworth e Chebyshev possuem características de fase mais pobres com o aumento da ordem.
