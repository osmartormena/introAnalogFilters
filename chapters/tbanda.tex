\chapter{Transformações de banda}

Em nossa motivação, baseada em circuitos RC simples, foi possível averiguar a resposta em frequência de filtros passa"-baixas, passa"-altas, passa"-faixa e rejeita"-faixa. Por simplicidade, ao abordar as aproximações de Butterworth, Chebyshev e Bessel, trabalhamos com protótipos passa"-baixas normalizados. No último Capítulo, fomos apresentados aos equacionamentos de ordem (e frequência de corte, se aplicável) a partir das especificações do filtro.

Neste Capítulo, daremos o último passo da síntese dos filtros: a desnormalização e transformação de banda. Nem sempre desejamos um filtro passa"-baixas e quase nunca temos utilidade para um normalizado.

\section{Desnormalização do protótipo passa"-baixas}

A desnormalização de um protótipo passa"-baixas consiste de um simples remapeamento em $s$ na função de transferência $H(s)$. Tomando o mapeamento
\begin{equation}
	s\mapsto\frac{s}{\omega_n},
\end{equation}
ele pode ser aplicado a um protótipo, como o Butterworth de ordem unitária:
\begin{align*}
	H(s)&=\frac{1}{s+1}\Bigg|_{s=\frac{s}{\omega_n}}=\frac{1}{\displaystyle\frac{s}{\omega_n}+1};\\
	H(s)&=\frac{\omega_n}{s+\omega_n},
\end{align*}
onde $\omega_n$ (em \unit{\radian\per\second}) será a nova frequência de corte do filtro passa"-baixas.

A função \lstinline{lp2lp()} do Singal Processing Toolbox realiza a desnormalização de um protótipo passa"-baixas.

\section{Conversão para passa"-altas}

O remapeamento em $s$ da função de transferência $H(s)$ para conversão entre um protótipo passa"-baixas normalizado para um filtro passa"-altas é dado por
\begin{equation}
	s\mapsto\frac{\omega_n}{s}.
\end{equation}

Novamente, para ilustrar, vamos transformar um protótipo Butterworth de ordem unitária:
\begin{align*}
	H(s)&=\frac{1}{s+1}\Bigg|_{s=\frac{\omega_n}{s}}=\frac{1}{\displaystyle\frac{\omega_n}{s}+1};\\
	H(s)&=\frac{s}{s+\omega_n},
\end{align*}
cuja forma é idêntica à função de transferência do circuito da \fig{Hhp}.

A função \lstinline{lp2hp()} do Singal Processing Toolbox realiza a transformação de banda para um filtro passa"-altas.

\section{Conversão para passa"-faixa}

O remapeamento em $s$ da função de transferência $H(s)$ para conversão entre um protótipo passa"-baixas normalizado para um filtro passa"-faixa é dado por
\begin{equation}
	s\mapsto\frac{\omega_n}{BW}\frac{s^2+1}{s},
\end{equation}
onde $\omega_n$, neste contexto, é a frequência central da banda, dada pela média geométrica dos limites da banda de passagem $\omega_{p,1}$ e $\omega_{p,2}$
\begin{equation}
	\omega_n=\sqrt{\omega_{p,1}\omega_{p,2}},
\end{equation}
enquanto $BW$ é a largura de banda, dada pela diferença entre $\omega_{p,2}$ e $\omega_{p,1}$
\begin{equation}
	BW=\omega_{p,2}-\omega_{p,1}.
\end{equation}

Mais uma vez, para ilustrar, vamos transformar um protótipo Butterworth de ordem unitária:
\begin{align*}
	H(s)&=\frac{1}{s+1}\Bigg|_{s=\frac{\omega_n}{BW}\frac{s^2+1}{s}}=\frac{1}{\displaystyle\frac{\omega_n}{BW}\frac{s^2+1}{s}+1};\\
	H(s)&=\frac{BWs}{\omega_ns^2+BWs+\omega_n}=\frac{\displaystyle\frac{BW}{\omega_n}s}{\displaystyle s^2+\frac{BW}{\omega_n}s+1},
\end{align*}
cuja forma é idêntica à função de transferência do circuito da \fig{Hbp}. Ressalto aqui que a grandeza $\omega_n\slash BW\approx Q$ para $Q$ suficientemente alto.

A função \lstinline{lp2bp()} do Singal Processing Toolbox realiza a transformação de banda para um filtro passa"-faixa.

\section{Conversão para rejeita"-faixa}

O remapeamento em $s$ da função de transferência $H(s)$ para conversão entre um protótipo passa"-baixas normalizado para um filtro rejeita"-faixa é dado por
\begin{equation}
	s\mapsto\frac{BW}{\omega_n}\frac{s}{s^2+1},
\end{equation}
onde $\omega_n$, neste contexto, é a frequência central da banda, dada pela média geométrica dos limites da banda de rejeição $\omega_{s,1}$ e $\omega_{s,2}$
\begin{equation}
	\omega_n=\sqrt{\omega_{s,1}\omega_{s,2}},
\end{equation}
enquanto $BW$ é a largura de banda, dada pela diferença entre $\omega_{s,2}$ e $\omega_{s,1}$
\begin{equation}
	BW=\omega_{s,2}-\omega_{s,1}.
\end{equation}

Uma última vez, para ilustrar, vamos transformar um protótipo Butterworth de ordem unitária:
\begin{align*}
	H(s)&=\frac{1}{s+1}\Bigg|_{s=\frac{BW}{\omega_n}\frac{s}{s^2+1}}=\frac{1}{\displaystyle\frac{BW}{\omega_n}\frac{s}{s^2+1}+1};\\
	H(s)&=\frac{\omega_n(s^2+1)}{\omega_ns^2+BWs+\omega_n}=\frac{s^2+1}{\displaystyle s^2+\frac{BW}{\omega_n}s+1},
\end{align*}
cuja forma é idêntica à função de transferência do circuito da \fig{Hsb}.

A função \lstinline{lp2bs()} do Singal Processing Toolbox realiza a transformação de banda para um filtro passa"-faixa.

\section{Síntese de filtros analógicos}

Um \eng{script} especializado no Matlab\textsuperscript{®} será disponibilizado para a síntese de filtros de Butterworth, Chebyshev e Bessel. Ele fará uso das funções \lstinline{butter()}, \lstinline{cheby1()} e \lstinline{besself()} do Signal Processing Toolbox.
